\documentclass[a4paper,12pt]{article}
\usepackage[utf8]{inputenc}
\usepackage{graphicx}
\usepackage{hyperref}
\usepackage{fancyhdr}
\usepackage{geometry}
\geometry{a4paper, margin=1in}
\usepackage{titlesec}
\usepackage{tocbibind}

% Imposta intestazione e piè di pagina
\pagestyle{fancy}
\fancyhead[L]{Illegal Gambling}
\fancyhead[R]{\thepage}
\fancyfoot{}

\title{\textbf{Illegal Gambling}}
\author{Azael Garcia Rufer - Riccardo Polazzi \\ Università degli Studi di Bologna \\ Ingegneria e Scienze Informatiche}
\date{\today}

\begin{document}

\maketitle

\begin{abstract}
Il gioco d'azzardo illegale è un fenomeno diffuso a livello globale, con implicazioni economiche, sociali e legali significative. Questa relazione analizza le principali caratteristiche del fenomeno, i suoi impatti e le normative esistenti per contrastarlo.
\end{abstract}

\newpage
\tableofcontents
\newpage

\section{Introduzione}

Il gioco d’azzardo illegale rappresenta una minaccia significativa per la società contemporanea, poiché non solo danneggia economicamente individui e famiglie, ma spesso è anche collegato ad attività criminali organizzate e problematiche sociali. L’espansione dei mercati del gioco, facilitata dalla deregolamentazione e dall’accesso digitale, ha aumentato il rischio di condotte illecite e comportamenti compulsivi, contribuendo alla percezione negativa del gioco nella popolazione generale. \ \ \textbf{Fonte:} \textit{A Taxonomy of Gambling Related Crime, pp. 3--4} \cite{banks2018taxonomy}

Sebbene alcuni studi si siano focalizzati sull’analisi dei benefici economici del gioco legale, crescenti evidenze indicano che le ricadute sociali del gioco d’azzardo patologico – come i reati per sostenere il comportamento di gioco e l’impatto sulle famiglie – costituiscono un problema di salute pubblica. \ \textbf{Fonti:} \textit{Gambling Disorder Related Illegal Acts, pp. 4--5} \cite{gorsane2017illegalacts}; \textit{Social Costs of Gambling Harm in Italy, pp. 72--80} \cite{lucchini2022socialcosts}

È stato osservato che le attività illegali collegate al gioco d’azzardo, come il riciclaggio di denaro e la frode, hanno profonde implicazioni sulla stabilità del sistema finanziario, coinvolgendo talvolta anche istituti bancari e istituzioni pubbliche. \ \textbf{Fonte:} \textit{Prevention of Criminal Acts of Money Laundering, p. 44} \cite{tarina2019moneylaundering}

Allo stesso tempo, gruppi vulnerabili come adolescenti e soggetti con dipendenze risultano particolarmente esposti, anche per la scarsa consapevolezza dei rischi e la facilità d’accesso a piattaforme non regolamentate. \ \textbf{Fonti:} \textit{Relationship Between Gambling Severity and Attitudes, pp. 2--4} \cite{gori2014adolescentgambling}; \textit{The Risk of Gambling Disorders in Children, pp. 3--5} \cite{ferrara2019childrenrisk}

Questa relazione intende esplorare il fenomeno del gioco d’azzardo illegale da una prospettiva multidisciplinare, analizzando le sue forme principali, l’impatto economico e sociale, i legami con la criminalità, la normativa vigente a livello internazionale e alcune strategie di intervento e prevenzione.


\section{Tipologie di Gioco d'Azzardo Illegale}

Il gioco d'azzardo illegale assume molteplici forme, spesso strettamente legate al contesto socioeconomico e al livello di regolamentazione presente nei vari Paesi. Esso si differenzia dal gioco legale per l’assenza di autorizzazioni statali, la mancanza di controlli fiscali e di tutela per i giocatori, e per il frequente coinvolgimento della criminalità organizzata. Le forme principali sono descritte di seguito.

\subsection{Scommesse clandestine}

Le scommesse clandestine sono una delle espressioni più diffuse del gioco d’azzardo illegale. Consistono nell’accettazione di puntate su eventi sportivi o altre manifestazioni (ad esempio corse ippiche o eventi politici) senza alcuna licenza ufficiale o autorizzazione amministrativa. In molti casi, tali attività sono gestite da organizzazioni criminali che le utilizzano come strumenti per il riciclaggio di denaro proveniente da attività illecite.

\textbf{Fonte:} \textit{Social Costs of Gambling Harm in Italy, p. 79} \cite{lucchini2022socialcosts}; \textit{A Taxonomy of Gambling Related Crime, pp. 5--7} \cite{banks2018taxonomy}

\subsection{Casinò non autorizzati}

I casinò non autorizzati operano senza alcuna licenza statale e rappresentano una forma particolarmente pericolosa di gioco d’azzardo illegale. Queste strutture possono assumere l’aspetto di locali notturni, circoli “culturali” o abitazioni private adibite stabilmente al gioco. In tali contesti, l’assenza di norme di sicurezza e di tutela dei consumatori espone i giocatori a rischi elevati.

Tali casinò non solo eludono completamente i controlli fiscali, ma offrono anche l’ambiente ideale per lo svolgimento di altre attività criminali come usura, spaccio di droga e riciclaggio di denaro.

\textbf{Fonte:} \textit{Prevention of Criminal Acts of Money Laundering, pp. 45--46} \cite{tarina2019moneylaundering}

\subsection{Giochi online non regolamentati}

Il gioco d’azzardo online costituisce una frontiera in rapida espansione dell’illegalità. Nonostante la presenza di sistemi di regolazione nei singoli Paesi, molti portali web operano da giurisdizioni estere dove la normativa è assente o estremamente blanda. Tali siti accettano scommesse da utenti italiani o europei senza avere le necessarie concessioni e senza rispettare i requisiti di trasparenza, prevenzione della dipendenza e tutela dei minori.

La facilità di accesso, unita all’apparente legalità dell’interfaccia utente, rende difficile distinguere tra portali autorizzati e illegali. In particolare, i giovani risultano particolarmente vulnerabili: secondo una ricerca italiana, il 53,5% dei minorenni ha già partecipato ad almeno un’attività di gioco, nonostante i divieti legislativi.

\textbf{Fonte:} \textit{Relationship Between Gambling Severity and Attitudes, pp. 3--4} \cite{gori2014adolescentgambling}; \textit{The Risk of Gambling Disorders in Children, pp. 4--5} \cite{ferrara2019childrenrisk}

\subsection{Lotterie illegali}

Le lotterie illegali rappresentano una forma storica e ancora attuale di gioco clandestino, spesso radicata in contesti di marginalità economica o sociale. Queste attività possono avvenire tramite la vendita di biglietti “fatti in casa” o la raccolta manuale di numeri e puntate, senza alcun sistema ufficiale di verifica o garanzia. In molti casi, i premi sono inferiori a quelli promessi, o non vengono mai erogati.

Un esempio emblematico è rappresentato dal Brasile, dove migliaia di videopoker illegali sequestrati dalla polizia sono stati poi riconvertiti in strumenti educativi da università e ONG, mostrando la vastità e diffusione di questi dispositivi sul territorio.

\textbf{Fonte:} \textit{From Cheating to Teaching, pp. 10--11} \cite{bento2010cheating}


\section{Impatto del Gioco d'Azzardo Illegale}

Il gioco d’azzardo illegale genera una vasta gamma di conseguenze negative che si estendono ben oltre il singolo giocatore, coinvolgendo famiglie, comunità locali, istituzioni pubbliche e sistemi economici. Gli impatti possono essere suddivisi in almeno quattro grandi categorie: economici, sociali, legati alla criminalità organizzata e sanitari. L’analisi di questi effetti consente di comprendere l’urgenza di strategie integrate per contrastare il fenomeno.

\subsection{Effetti economici}

L’impatto economico del gioco d’azzardo illegale è particolarmente rilevante in termini di evasione fiscale, perdite per lo Stato e distorsione del mercato regolamentato. Secondo uno studio condotto in Italia, i costi sociali associati al gioco patologico superano i 2,3 miliardi di euro l’anno. Tali stime includono costi diretti come trattamenti sanitari, interventi sociali e spese legali, ma anche costi indiretti legati alla perdita di produttività, disoccupazione e impatti sulle famiglie \textit{(Social Costs of Gambling Harm in Italy, pp. 72--73)} \cite{lucchini2022socialcosts}.

Tuttavia, questa cifra rappresenta un valore fortemente sottostimato, poiché non include i costi associati ai giocatori a basso e medio rischio, né quelli sostenuti dagli “altri significativi” (familiari, amici, colleghi), che spesso subiscono anch’essi ripercussioni economiche e psicologiche.

\subsection{Conseguenze sociali}

Il gioco d’azzardo illegale è associato a un marcato deterioramento delle relazioni personali, familiari e lavorative. Nei casi più gravi, si osservano rotture familiari, perdita della custodia dei figli, isolamento sociale e indebitamento estremo. In ambito adolescenziale, è stato rilevato che più della metà dei minori italiani (53,5%) ha partecipato almeno una volta ad attività di gioco, nonostante i divieti legislativi. Questo dato è particolarmente preoccupante in quanto il gioco in giovane età è correlato a una maggiore probabilità di sviluppare comportamenti patologici in età adulta \textit{(Relationship Between Gambling Severity and Attitudes, pp. 3--4)} \cite{gori2014adolescentgambling}.

Inoltre, la diffusione del gioco non regolamentato compromette il senso di legalità nella società, specialmente in contesti economicamente svantaggiati, dove la promessa di guadagni facili rappresenta una tentazione concreta e pericolosa. I danni sociali si estendono anche al sistema educativo e al tessuto associativo, indebolendo il capitale sociale delle comunità colpite.

\subsection{Collegamenti con la criminalità organizzata}

Molti studi evidenziano il legame strutturale tra gioco d’azzardo illegale e organizzazioni criminali. In particolare, le attività illecite nel settore del gioco costituiscono un importante canale per il riciclaggio di denaro proveniente da traffico di droga, estorsione e corruzione. La possibilità di inserire somme di denaro in circuiti “puliti”, ad esempio attraverso scommesse fittizie o false vincite, rende il settore particolarmente attraente per le mafie \textit{(A Taxonomy of Gambling Related Crime, pp. 5--7)} \cite{banks2018taxonomy}; \textit{(Prevention of Criminal Acts of Money Laundering, pp. 45--46)} \cite{tarina2019moneylaundering}.

Inoltre, in contesti dove la regolamentazione è debole o frammentata, le organizzazioni criminali riescono a operare indisturbate, controllando intere filiere di scommesse e sale giochi non autorizzate, e guadagnando consenso sociale grazie alla redistribuzione illecita della ricchezza.

\subsection{Rischi per i giocatori}

I giocatori che si affidano a circuiti illegali risultano maggiormente esposti a comportamenti compulsivi, truffe e mancanza di supporto. Nei casi più estremi, il comportamento di gioco sfocia in atti illegali (furto, frode, appropriazione indebita) per reperire denaro, soprattutto quando il soggetto è affetto da disturbo da gioco d’azzardo (Gambling Disorder) \textit{(Gambling Disorder Related Illegal Acts, pp. 4--6)} \cite{gorsane2017illegalacts}.

Uno studio clinico ha evidenziato che i giocatori con comportamenti illegali associati al gioco tendono ad avere punteggi significativamente più alti nella South Oaks Gambling Scale (SOGS), maggiore presenza di comorbidità psichiatriche e un livello di reddito più basso rispetto alla media dei giocatori patologici. Tali soggetti rappresentano un gruppo ad alto rischio anche per la recidiva e la scarsa risposta ai trattamenti.

Nei minori, il rischio è amplificato dalla bassa percezione del pericolo: molti adolescenti non riconoscono il gioco come attività potenzialmente dannosa e tendono a sottovalutarne l’impatto, specialmente quando frequentano gruppi di pari che giocano abitualmente \textit{(The Risk of Gambling Disorders in Children, pp. 4--5)} \cite{ferrara2019childrenrisk}.


\section{Normative e Legislazione Internazionale}

La regolamentazione del gioco d’azzardo illegale varia ampiamente a livello globale, rispecchiando differenti approcci culturali, giuridici ed economici. Alcuni Paesi adottano normative permissive volte a incanalare il gioco entro circuiti legali e controllati, mentre altri optano per un approccio proibizionista che, però, può favorire la nascita di mercati clandestini. In questa sezione verranno analizzate le principali esperienze normative negli Stati Uniti, in Europa, in Asia e il ruolo degli organismi internazionali.

\subsection{Regolamentazioni negli Stati Uniti}

Negli Stati Uniti, la regolamentazione del gioco d’azzardo è altamente frammentata. Ogni Stato ha la facoltà di decidere quali forme di gioco autorizzare e in quali contesti. Questo ha portato alla coesistenza di mercati legali molto sviluppati – come quelli del Nevada e del New Jersey – con altri Stati che mantengono forti restrizioni. Tuttavia, anche nei contesti più permissivi, le attività illecite continuano a proliferare.

Secondo uno studio, le organizzazioni criminali negli Stati Uniti sfruttano le lacune normative per gestire operazioni parallele a quelle legali, facilitando il riciclaggio di denaro e la manipolazione di eventi sportivi tramite scommesse clandestine. \textit{(cfr. \textbf{A Taxonomy of Gambling Related Crime}, pp. 5--7)} \cite{banks2018taxonomy}

\subsection{Normative Europee}

In Europa, la regolamentazione del gioco d’azzardo presenta anch’essa un elevato grado di eterogeneità. Paesi come l’Italia, la Francia e la Germania hanno adottato un approccio centralizzato per gestire l’offerta di gioco attraverso concessioni pubbliche, mentre altri, come il Regno Unito, hanno liberalizzato il settore sotto la supervisione di autorità indipendenti.

Particolare attenzione è stata dedicata alla regolamentazione del gioco online: in Germania, ad esempio, il Trattato Interstatale sul Gioco ha limitato drasticamente l’accesso degli operatori esteri al mercato nazionale, sollevando però critiche per il rischio di migrazione degli utenti verso siti non autorizzati. In Francia, l’Autorité Nationale des Jeux ha introdotto misure stringenti, tra cui blocchi IP, multe e pene detentive, per contrastare l’offerta di giochi da parte di oltre 550 operatori non autorizzati. \textit{(cfr. \textbf{A Taxonomy of Gambling Related Crime}, pp. 8--10)} \cite{banks2018taxonomy}

\subsection{Legislazione in Asia e altre regioni}

In Asia, il gioco d’azzardo è spesso formalmente vietato, ma praticato su larga scala sia in forma clandestina che attraverso casinò legali infiltrati da gruppi mafiosi. Il caso di Macao è emblematico: studi recenti hanno mostrato come la criminalità organizzata operi quotidianamente nei casinò locali, coinvolgendosi in attività come riciclaggio, truffe e corruzione – il tutto con una struttura che ricorda quella di un istituto finanziario, piuttosto che un’organizzazione violenta. \textit{(cfr. \textbf{A Taxonomy of Gambling Related Crime}, pp. 10--11)} \cite{banks2018taxonomy}

Anche in altri contesti, come alcune province cinesi e regioni del Sud-est asiatico, l’assenza di un quadro normativo chiaro favorisce lo sviluppo di un’economia parallela legata al gioco illegale.

\subsection{Ruolo delle organizzazioni internazionali}

Le organizzazioni internazionali svolgono un ruolo sempre più rilevante nel promuovere la cooperazione tra Stati per il contrasto al gioco illegale. La Commissione Europea ha adottato una serie di raccomandazioni finalizzate all’armonizzazione delle regole, alla protezione dei consumatori e alla lotta contro il riciclaggio di denaro. \textit{(cfr. \textbf{Social Costs of Gambling Harm in Italy}, p. 72)} \cite{lucchini2022socialcosts}

Anche le Nazioni Unite e l’Interpol hanno inserito il gioco d’azzardo tra le attività monitorate nell’ambito della criminalità economica transnazionale, sottolineando la necessità di sistemi informativi integrati e operazioni coordinate su scala sovranazionale. \textit{(cfr. \textbf{A Taxonomy of Gambling Related Crime}, p. 11)} \cite{banks2018taxonomy}

\section{Caso di Studio}

L’analisi di casi concreti consente di comprendere la complessità e le molteplici sfaccettature del gioco d’azzardo illegale. Di seguito vengono presentati due esempi emblematici: uno italiano e uno internazionale, entrambi rappresentativi delle strategie adottate per il contrasto al fenomeno.

\subsection{Operazione italiana contro le scommesse illegali}

In Italia, numerose operazioni giudiziarie hanno messo in luce l’estensione delle reti di scommesse clandestine legate alla criminalità organizzata. Le autorità italiane hanno documentato la presenza di circuiti paralleli che utilizzano piattaforme online non autorizzate, talvolta con sede all’estero, per raccogliere scommesse in violazione della normativa nazionale.

Un esempio emblematico riguarda le indagini condotte in collaborazione con l’Agenzia delle Dogane e dei Monopoli, in cui sono state sequestrate centinaia di terminali non autorizzati, spesso installati in locali pubblici come bar e circoli ricreativi. Le indagini hanno evidenziato che tali attività erano utilizzate non solo per eludere il fisco, ma anche per riciclare ingenti somme di denaro provenienti da traffici illeciti \textit{(cfr. \textbf{Social Costs of Gambling Harm in Italy}, p. 79)}:contentReference[oaicite:0]{index=0}.

\subsection{Inchiesta internazionale: il caso delle videolottery in Brasile}

Un altro caso rilevante si è verificato in Brasile, dove migliaia di macchine da gioco illegali (videopoker e slot) sono state sequestrate dalla polizia in operazioni coordinate. Questi dispositivi, inizialmente destinati alla distruzione, sono stati in parte riconvertiti in strumenti educativi grazie a un progetto universitario promosso dalla Southern University of Santa Catarina.

Le schede elettroniche, i monitor e le componenti dei videopoker sono state riutilizzate per assemblare computer da destinare a scuole pubbliche, centri giovanili e programmi di inclusione digitale per bambini e ragazzi con bisogni educativi speciali. Il progetto ha permesso di trasformare un simbolo di devianza e illegalità in una risorsa per la comunità locale \textit{(cfr. \textbf{From Cheating to Teaching}, pp. 10--11)}:contentReference[oaicite:1]{index=1}.

Questo caso dimostra come, oltre alla repressione, possano essere adottate strategie di riutilizzo sociale che uniscono legalità, innovazione tecnologica e inclusione.


\section{Strategie di Prevenzione e Controllo}

Contrastare il gioco d’azzardo illegale richiede un approccio multidimensionale che combini interventi tecnologici, cooperazione internazionale, programmi educativi e strategie di riduzione del danno. Di seguito vengono analizzate alcune delle principali modalità di intervento proposte o attuate a livello internazionale.

\subsection{Tecnologie per il monitoraggio delle transazioni}

L’utilizzo della tecnologia rappresenta uno strumento chiave per individuare e contrastare le attività di gioco illegale. Le autorità possono implementare sistemi informatici per il tracciamento delle transazioni sospette, in particolare nei punti vendita, nei siti web e nei flussi bancari. Il monitoraggio automatizzato delle transazioni consente di rilevare comportamenti anomali, ad esempio flussi di denaro irregolari o movimenti di fondi in aree ad alto rischio.

In Italia, l’Agenzia delle Dogane e dei Monopoli ha già introdotto piattaforme per la rilevazione di apparecchi non autorizzati e anomalie nei flussi di gioco. Tali sistemi sono descritti come strumenti centrali nella prevenzione dell’evasione fiscale e del riciclaggio, secondo quanto riportato da Lucchini e Comi nel loro studio sui costi sociali del gambling in Italia \textit{(Social Costs of Gambling Harm in Italy, p. 79)} \cite{lucchini2022socialcosts}.

\subsection{Collaborazione tra governi e piattaforme digitali}

Un altro elemento essenziale è la collaborazione tra enti governativi, autorità regolatrici e piattaforme digitali. La natura globale del gioco d’azzardo online impone la creazione di canali di comunicazione e cooperazione tra diversi Paesi per contrastare i siti non autorizzati e promuovere il gioco legale.

Lo studio di Lucchini e Comi evidenzia la necessità di coinvolgere anche banche e gestori di pagamenti digitali per bloccare le transazioni verso operatori non conformi e creare black-list condivise tra le autorità \textit{(Social Costs of Gambling Harm in Italy, pp. 79--80)} \cite{lucchini2022socialcosts}.

\subsection{Educazione e sensibilizzazione}

La prevenzione passa anche attraverso l’informazione e la sensibilizzazione. Programmi educativi rivolti a giovani, famiglie e scuole possono ridurre l’incidenza del gioco problematico e migliorare la consapevolezza dei rischi associati all’azzardo.

Nel loro studio su 14.910 studenti italiani, Gori et al. sottolineano che il 53,5\% dei minorenni ha già sperimentato il gioco d’azzardo, evidenziando un urgente bisogno di educazione specifica. Gli autori raccomandano l’uso di strumenti integrati e campagne di comunicazione basate su evidenze scientifiche \textit{(Relationship Between Gambling Severity and Attitudes, pp. 3--4)} \cite{gori2014adolescentgambling}.

\subsection{Politiche di riduzione del danno}

Le politiche di riduzione del danno mirano a contenere gli effetti negativi del gioco patologico anche in assenza di completa astinenza. Tali politiche includono l’introduzione di limiti di spesa, autoesclusione dai siti di gioco, supporto psicologico accessibile e reti territoriali di intervento rapido.

Lucchini e Comi propongono un approccio di salute pubblica che integri aspetti clinici e sociali, ponendo l’accento sull'importanza di servizi accessibili e non stigmatizzanti \textit{(Social Costs of Gambling Harm in Italy, pp. 77--80)} \cite{lucchini2022socialcosts}.

\section{Conclusioni}

Il gioco d’azzardo illegale costituisce una sfida complessa e multidimensionale che coinvolge aspetti economici, sociali, sanitari e giuridici. La sua diffusione è alimentata da numerosi fattori, tra cui la deregolamentazione digitale, l’interesse della criminalità organizzata, la fragilità di alcune fasce della popolazione e la difficoltà di controllo da parte delle autorità.

Dalla letteratura analizzata emerge chiaramente che il gioco illegale genera costi sociali ingenti. In Italia, tali costi sono stimati in oltre 2,3 miliardi di euro annui, includendo spese per trattamenti sanitari, perdita di produttività, disoccupazione e interventi giudiziari \textit{(Social Costs of Gambling Harm in Italy, pp. 72--80)} \cite{lucchini2022socialcosts}. A ciò si aggiunge l’impatto sulla salute mentale e relazionale dei giocatori, spesso aggravato dall’assenza di servizi dedicati e dalla stigmatizzazione sociale \textit{(Gambling Disorder Related Illegal Acts, pp. 4--6)} \cite{gorsane2017illegalacts}.

Particolarmente allarmante è il dato relativo all’accesso al gioco da parte degli adolescenti: più della metà dei minorenni ha avuto esperienze di gioco, e una parte significativa mostra comportamenti a rischio o già problematici \textit{(Relationship Between Gambling Severity and Attitudes, pp. 2--4)} \cite{gori2014adolescentgambling}. Questo richiede un intervento educativo e preventivo urgente, mirato non solo alla conoscenza del fenomeno, ma anche allo sviluppo di competenze psicosociali.

Le normative internazionali, pur presentando una forte eterogeneità, convergono su alcuni elementi chiave: la necessità di regolamentare il gioco online, rafforzare i controlli sui flussi finanziari e favorire la cooperazione tra Stati \textit{(A Taxonomy of Gambling Related Crime, pp. 5--11)} \cite{banks2018taxonomy}. Tuttavia, è evidente che l’approccio repressivo da solo non è sufficiente.

In conclusione, il contrasto al gioco d’azzardo illegale deve passare da una strategia integrata che includa:

\begin{itemize}
    \item l’adozione di tecnologie per il monitoraggio delle transazioni e il blocco delle piattaforme illegali;
    \item la promozione di campagne di educazione e sensibilizzazione nelle scuole e nelle comunità;
    \item il potenziamento delle reti di assistenza sanitaria e sociale, con servizi di bassa soglia e accesso facilitato;
    \item la cooperazione transnazionale tra governi, autorità di controllo e gestori di pagamento;
    \item il coinvolgimento attivo della società civile nella promozione di legalità e responsabilità.
\end{itemize}

Solo attraverso un approccio sistemico e partecipato sarà possibile ridurre l’impatto del gioco d’azzardo illegale e tutelare i soggetti più vulnerabili.


\begin{thebibliography}{99}

\bibitem{banks2018taxonomy}
Banks, J., \& Waugh, D. (2018). \textit{A Taxonomy of Gambling Related Crime}. International Gambling Studies. 
(PDF: \texttt{A\_taxonomy\_of\_gambling\_related\_crime.pdf})

\bibitem{bento2010cheating}
Bento da Silva, J., Alves, G. R., \& Mota Alves, J. B. (2010). \textit{From Cheating to Teaching: A Path for Conversion of Illegal Gambling Machines}. eLearning Papers, 19.
(PDF: \texttt{From\_cheating\_to\_teaching\_a\_path\_for\_con.pdf})

\bibitem{gorsane2017illegalacts}
Gorsane, M. A., et al. (2017). \textit{Gambling disorder-related illegal acts: Regression model of associated factors}. Journal of Behavioral Addictions, 6(1), 64–73.
(PDF: \texttt{Gambling\_disorder\_related\_illegal\_acts\_R.pdf})

\bibitem{tarina2019moneylaundering}
Tarina, D. D. Y., Dinanti, D., \& Sakti, M. (2019). \textit{Prevention of Criminal Acts of Money Laundering in Banks}. Asia Pacific Fraud Journal, 4(1), 44–59.
(PDF: \texttt{Prevention\_of\_Criminal\_Acts\_of\_Money\_Lau.pdf})

\bibitem{matilainen2017finland}
Matilainen, R. (2017). \textit{Production and Consumption of Recreational Gambling in Twentieth-Century Finland}. Doctoral Dissertation, University of Helsinki.
(PDF: \texttt{Production\_and\_consumption\_of\_recreation.pdf})

\bibitem{gori2014adolescentgambling}
Gori, M., et al. (2014). \textit{Relationship Between Gambling Severity and Attitudes in Adolescents: Findings from a Population-Based Study}. Journal of Gambling Studies.
(PDF: \texttt{Relationship\_Between\_Gambling\_Severity\_a.pdf})

\bibitem{lucchini2022socialcosts}
Lucchini, F., \& Comi, S. L. (2022). \textit{Social Costs of Gambling Harm in Italy}. Critical Gambling Studies, 3(1), 71–82. 
(PDF: \texttt{Social\_Costs\_of\_Gambling\_Harm\_in\_Italy.pdf})

\bibitem{ferrara2019childrenrisk}
Ferrara, P., et al. (2019). \textit{The Risk of Gambling Disorders in Children and Adolescents}. The Journal of Pediatrics.
(PDF: \texttt{The\_Risk\_of\_Gambling\_Disorders\_in\_Childr.pdf})

\end{thebibliography}

\end{document}