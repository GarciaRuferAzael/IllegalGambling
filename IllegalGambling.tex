\documentclass[a4paper,12pt]{article}
\usepackage[utf8]{inputenc}
\usepackage{graphicx}
\usepackage{hyperref}
\usepackage{fancyhdr}
\usepackage{geometry}
\geometry{a4paper, margin=1in}
\usepackage{titlesec}
\usepackage{tocbibind}

% Imposta intestazione e piè di pagina
\pagestyle{fancy}
\fancyhead[L]{Illegal Gambling}
\fancyhead[R]{\thepage}
\fancyfoot{}

\title{\textbf{Illegal Gambling}}
\author{Azael Garcia Rufer - Riccardo Polazzi \\ Università degli Studi di Bologna \\ Ingegneria e Scienze Informatiche}
\date{\today}

\begin{document}

\maketitle

\begin{abstract}
Il gioco d'azzardo illegale è un fenomeno diffuso a livello globale, con implicazioni economiche, sociali e legali significative. Questa relazione analizza le principali caratteristiche del fenomeno, i suoi impatti e le normative esistenti per contrastarlo.
\end{abstract}

\newpage
\tableofcontents
\newpage

\section{Introduzione}
Il gioco d'azzardo illegale rappresenta un problema complesso che coinvolge diversi attori, dalle organizzazioni criminali ai giocatori stessi. Questo documento esplora la natura del fenomeno, le sue cause principali e le conseguenze per la società.

\section{Tipologie di Gioco d'Azzardo Illegale}
\subsection{Scommesse clandestine}
\subsection{Casinò non autorizzati}
\subsection{Giochi online non regolamentati}
\subsection{Lotterie illegali}

\section{Impatto del Gioco d'Azzardo Illegale}
\subsection{Effetti economici}
\subsection{Conseguenze sociali}
\subsection{Collegamenti con la criminalità organizzata}
\subsection{Rischi per i giocatori}

\section{Normative e Legislazione Internazionale}
\subsection{Regolamentazioni negli Stati Uniti}
\subsection{Normative Europee}
\subsection{Legislazione in Asia e altre regioni}
\subsection{Ruolo delle organizzazioni internazionali}

\section{Strategie di Prevenzione e Controllo}
\subsection{Tecnologie per il monitoraggio delle transazioni}
\subsection{Collaborazione tra governi e piattaforme digitali}
\subsection{Educazione e sensibilizzazione}
\subsection{Politiche di riduzione del danno}

\section{Conclusioni}
Il contrasto al gioco d'azzardo illegale richiede un approccio multidisciplinare che coinvolga governi, forze dell'ordine e istituzioni finanziarie. L'adozione di normative più severe e l'utilizzo di strumenti tecnologici avanzati possono contribuire a ridurre l'impatto di questo fenomeno.

\section{Riferimenti}
\begin{thebibliography}{9}
    \bibitem{who} World Health Organization. (2022). Report on Gambling Disorders.
    \bibitem{eu} European Commission. (2023). Online Gambling Regulations.
    \bibitem{us} U.S. Department of Justice. (2021). Illegal Gambling Enforcement Policies.
    \bibitem{un} United Nations Office on Drugs and Crime. (2023). Organized Crime and Gambling.
\end{thebibliography}

\end{document}